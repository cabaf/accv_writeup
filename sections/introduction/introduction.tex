\section{Introduction}
\label{introduction}

Human action recognition is a challenging task for computer vision algorithms due to the large variabilities in video data caused by occlusions,
camera motions, actor and scene appearances, among others. A popular current trend in action recognition methods relies on using local video
descriptors to represent visual events in videos \cite{Dollar2005, Laptev2005, WangCVPR2011}. These features are usually aggregated into
a compact representation, most commonly into a bag-of-features (BoF) representation framework \cite{Schuldt2004}. The advantage of this
simple representation is that it avoids difficult pre-processing steps such as motion segmentation and tracking.
In the BoF representation, local descriptors are quantized using a pre-computed codebook of visual patterns. This representation combined
with discriminative classifiers such as support vector machines (SVM), has achieved tremendous success in action recognition in controlled
scenarios \cite{Blank2005,Schuldt2004}. Due to its simplicity, BoF requires the use of strong, robust and informative features, which can be 
obtained reliably in such simplified scenarios. However, recent efforts in the collection of more realistic datasets from movies and web site 
\cite{Kuehne2011,Marszalek2009,Rodriguez2008} represent a challenge for existing methods due to dynamic backgrounds, changes in light 
conditions and camera motions among other noisy conditions. 

