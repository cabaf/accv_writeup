\section{Experimental results}
\label{results}
\subsection{Datasets and evaluation protocol}
\label{subsec:datasets}
We use four public datasets \cite{kuehne2011, marszalek2009, niebles2010, reddy2013} and their corresponding evaluation protocols. In this section we briefly describe each dataset.

\textbf{HMDB51} \cite{kuehne2011} includes a large collection of human activities categorized on 51 classes. It collects 6766 videos from different media resources \ie digitized movies, public databases and user generated web video data. Due to a large amount of videos contains undesired camera motions, the authors provide a stabilized version of the dataset. However, since we look at the camera motion as an informative cue, non-stabilized version of the dataset is used. For evaluating performance, we adopt the same protocol proposed by the dataset authors \ie computing the mean accuracy under three fixed train/test splits.

\textbf{Hollywood2} \cite{marszalek2009} contains a wide number of videos retrieved from 69 different Hollywood movies. It is divided in 12 categories including short actions such as Kiss, Answer Phone and Stand Up. This dataset remains as one of the most challenging despite the small number of action classes. Change of camera view,  camera motion and unchoreographed execution introduces more difficult a the time of recognition. To evaluate performance, we follow the author's protocol where videos are separated in two different sets: a training set of 823 videos and a testing set of 884 videos. We use training videos to learn our action models and then compute the mean average precision (mAP) over all action classes.

\textbf{Olympic Sports} \cite{niebles2010} or \textbf\textit{{Olympic}} comprises a set of 783 sport related YouTube videos. This set of videos are semi-automatically labeled using Amazon Mechanical Turk. This dataset establish new challenges for recognition because of it jumps from simple actions (\eg Kiss) to complex actions (\eg Hammer throw). All of these complex actions are related with olympic sports including actions like \textit{Long jump}, \textit{Pole vault} and \textit{Javelin throw}. As proposed by the author's dataset, we measure performance calculating the mAP over all dataset categories.

\textbf{UCF50} \cite{reddy2013} includes 6618 videos of 50 different human actions.  This dataset presents several recognition challenges due to large variations in camera motion, cluttered background, viewpoint, etc. Action categories are grouped into 25 sets, where each set consists of more than 4 action clips. Recognition performance is measured by applying a leave-one-group-out cross-validation and average accuracy over all group splits is reported. 


\subsection{Impact of contextual features}
We conduct further experiments to measure the  contribution of our proposed camera movement (CamMotion) and surrounding scene appearance descriptor (SIFT). Our Baseline corresponds to using only Foreground features for describing actions. Per-descriptor performances are compared to that established baseline. Also, we investigate the effect of combining proposed features with Foreground cues. As well, CamMotion and SIFT performance is evaluated under two action recognition representations \ie Bag of Features and Fisher vectors. Below, we present an analysis of obtained results.\\\\
\textbf{Representation}. As suggested in recent works \cite{perronnin2010, wang2013, xwang2013} Fisher vectors provides a boosted performance compared to traditional Bag of Feature representations. We found in our experiments that Fisher vectors also boost our contextual descriptors performance, as presented in Table \ref{tab:features}. However, we note that using Fisher vectors is less important with our CamMotion descriptor due to its low dimensionality. Even so, Fisher vectors are used for following analysis.\\\\
\textbf{Foreground-background}. As described in Section \ref{scene}, we perform a weak separation between background and foreground feature points. We measure the effect on performance of this separation in proposed features. We note that this type of weak segmentation provides a significant boost in performance, as observed in Table \ref{tab:segmentation}. When feature points are localized on the background, surrounding SIFT focuses on the scene appearance avoiding information of actors and foreground objects. The gain of surrounding SIFT over all the holistic approach is as follow: +0.3\% for HMDB51, +4.2\% for Hollywood2, +5.2\% for Olympics and +3.9\% for UCF50. The same behavior is observed with our CamMotion descriptor where performance is boosted in all datasets due to Fundamental Matrix is computed based on background tracks.\\\\
\textbf{Surrounding appearance}. While by itself SIFT achieves a discrete performance, it produces notable improvements when combined with foreground descriptors. As Table \ref{tab:features} reports, performance is significantly improved over all datasets. Interestingly, we note that surrounding SIFT produces higher improvements in HMDB51 and UCF50 \ie +2.7\% and +2.4\% respectively.\\\\
\textbf{Camera movement}. Experiment results evidences that action recognition is noticeably improved when a global motion is incorporated to Foreground features. Our CamMotion provides slightly lower contributions in performance than the SIFT descriptor. We observe a significantly contribution over all datasets except on HMDB51 where recognition performance decrease. This negative effect is attributed to the extensive shared shaky camera in video sequences for this dataset. This unable our CamMotion to capture discriminative cues over action categories. 

%%%%%%%%%%%%% Table: Effect of selected points %%%%%%%%%%%%%
\begin{table*}
\caption{Effect of separating background feature points surrounding SIFT and CamMotion. Experimental results consistently show that SIFT exhibit better results when is capturing the surrounding appearance of actions. Conversely, CamMotion and SIFT tend to be more discriminative when computed in non-foreground feature points.}
\begin{center}
{
\begin{tabular}{|c|c c|c c c c|}
\hline
& \multicolumn{2}{|c|}{Feature points} & \multicolumn{4}{|c|}{Datasets} \\
Feature $\downarrow$ & Foreground & Background & HMDB51 & Hollywood2 & Olympics & UCF50 \\
\hline
SIFT & \checkmark & & 19.5\% & 22.1\% & 33.5\% & 44.7\% \\
SIFT & & \checkmark & \textbf{20.1\%} & \textbf{28.5\%} & \textbf{39.6\%} & \textbf{49.8\%} \\
SIFT & \checkmark & \checkmark & 19.8\% & 24.3\% & 34.4\% & 45.9\% \\
\hline
CamMotion & \checkmark & & 9.7\% & 14.9\% & 19.5\% & 13.7\% \\
CamMotion & & \checkmark & \textbf{14.1\%} & \textbf{22.1\%} & \textbf{27.2\%} & \textbf{19.5\%} \\
CamMotion & \checkmark & \checkmark & 12.9\% & 18.7\% & 21.8\% & 17.2\% \\
\hline
\end{tabular}
}
\end{center}
\label{tab:segmentation}
\end{table*}
%%%%%%%%%%%%%%%%%%%%%%%%%%%%%%%%%%%%%%%%%%%%%%%%%%%%%%%%%%%%

%%%%%%%%%%%%%% Table: Feature analysis %%%%%%%%%%%%%%%%%%%%%
\begin{table*}
\caption{Impact of our surrounding scene appearance and camera movement in recognition performance. Bag of Features generally performs poor than Fisher vectors. Both surrounding SIFT and CamMotion show important improvements in performance when they are combined with foreground descriptors.}
\begin{center}
{
\def\arraystretch{1.11}
\setlength{\tabcolsep}{3.66pt}
\begin{tabular}{ |c c c|c c c c| }
\hline
\multicolumn{3}{|c|}{Features} & \multicolumn{4}{|c|}{Datasets} \\
Foreground & SIFT & CamMotion & HMDB51 & Hollywood2 & Olympics & UCF50 \\
\hline
\multicolumn{7}{|c|}{Framework: Bag of Features} \\
\hline
\checkmark & & & 51.2\% & 60.1\% & 79.8\% & 85.9\% \\
& \checkmark & & 19.5\% & 28.7\% & 36.4\% & 45.7\% \\
& & \checkmark & 13.5\% & 21.8\% & 26.9\% & 19.3\% \\
\checkmark & \checkmark & & 53.8\% & 60.9\% & 81.1\% & 87.2\% \\
\checkmark &  & \checkmark & 50.9\% & 60.4\% & 80.6\% & 86.8\% \\
& \checkmark & \checkmark & 20.7\% & 36.2\% & 43.7\% & 50.3\% \\
\checkmark & \checkmark & \checkmark & 51.7\% & 61.6\% & 81.7\% & 87.6\% \\
\hline
\multicolumn{7}{|c|}{Framework: Fisher vectors} \\
\hline
\checkmark & & & 56.5\% & 62.4\% & 90.4\% & 90.9\% \\
& \checkmark & & 20.1\% & 28.5\% & 39.6\% & 49.8\% \\
& & \checkmark & 14.1\% & 22.1\% & 27.2\% & 19.5\% \\
\checkmark & \checkmark & & \textbf{59.2\%} & \textbf{63.5\%} & \textbf{91.6\%} & \textbf{93.3\%} \\
\checkmark &  & \checkmark & 55.9\% & 62.9\% & 91.3\% & 93.1\% \\
& \checkmark & \checkmark & 22.3\% & 36.5\% & 46.5\% & 54.3\% \\
\checkmark & \checkmark & \checkmark & \textbf{57.9\%} & \textbf{64.1\%} & \textbf{92.5\%} & \textbf{93.8\%} \\
\hline
\end{tabular}
}
\end{center}
\label{tab:features}
\end{table*}
%%%%%%%%%%%%%%%%%%%%%%%%%%%%%%%%%%%%%%%%%%%%%%%%%%%%%%%%%%%%%


\subsection{Comparison with the state of the art}
We set side by side our method with recent methods that address the same application using similar representations, \ie methods that use dense trajectory points to represent video sequences \cite{wang2013, jiang2012, jain2013} in Table \ref{tab:stateofart}. We also present results for our own implementation of \cite{wang2013}, which correspond to our baseline (Foreground). The gain over the recent paper \cite{wang2013}, which reports the best performance in the literature, is as follow: \textbf{+2\%} for HMDB51, \textbf{+1.4\%} for Olympic Sports and \textbf{2.6\%} for UCF50. We also achieve a comparable performance on Hollywood2 dataset with only 0.2\% less in the mAP value. Since Human Detection (HD) is not included in our trajectory extraction stage, a more direct comparison its the non-HD approach of Wang \etal \cite{wang2013}. In that case, our method outperforms their improved trajectories in \textbf{3.3\%} for HMDB51, \textbf{1.1\%} for Hollywood2, \textbf{2.3\%} for Olympic Sports and \textbf{3.3\%} for UCF50.

%%%%%%%%%%%%%% Table: State-of-the-art %%%%%%%%%%%%%%%%%%%%%%
\begin{table*}
\caption{Comparison with the state-of-the-art on challenging datasets. Our method improves reported results in the state-of-the-art for three different datasets, HMDB51, Olympic Sports and UCF50 and obtains competitive peformance in Hollywod2.}
\begin{center}
{
\begin{tabular}{ |l| c c c c| }
\hline
Approach $\downarrow$ & HMDB51 & Hollywood2 & Olympics & UCF50 \\
\hline
Jiang \etal \cite{jiang2012} & 40.7\% & 59.5\% & 80.6 & - \\
Jain \etal \cite{jain2013} & 52.1\% & 62.5\% & 83.2 & - \\
Wang \etal \cite{wang2013} non-HD & 55.9\% & 63.0\% & 90.2\% & 90.5\% \\
Wang \etal \cite{wang2013} HD & 57.2\% & \textbf{64.3\%} & 91.1\% & 91.2\% \\
\hline
\multicolumn{5}{|c|}{\textit{Our methods with Fisher vectors}} \\
\hline
Baseline (Foreground) & 56.5\% & 62.4\% & 90.4\% & 90.9\% \\
Foreground + SIFT & \textbf{59.2\%} & 63.5\% & \textbf{91.6\%} & \textbf{93.3\%} \\
Foreground + SIFT + CamMotion  & \textbf{57.9\%} & \textbf{64.1\%} & \textbf{92.5\%} & \textbf{93.8\%} \\
\hline
\end{tabular}
}
\end{center}
\label{tab:stateofart}
\end{table*}
%%%%%%%%%%%%%%%%%%%%%%%%%%%%%%%%%%%%%%%%%%%%%%%%%%%%%%%%%%%%%
