% Updated in May 2014 by Hideo Saito
% Updated in March 2012 by Yasuyuki Matsushita
% Updated in April 2002 by Antje Endemann, ...., and in March 2010 by Reinhard Klette
% Based on CVPR 07 and LNCS style, with modifications by DAF, AZ and elle 2008, AA 2010, ACCV 2010

\documentclass[runningheads]{llncs}
\usepackage{graphicx}
\usepackage{amsmath,amssymb} % define this before the line numbering.
\usepackage{ruler}
\usepackage{color}
\usepackage{cite}

%===========================================================
% Commands

%===========================================================
\begin{document}
\pagestyle{headings}
\mainmatter

\def\ACCV14SubNumber{165}  % Insert your submission number here

%===========================================================
\title{Surrounding cues for human action recognition} % Replace with your title
\titlerunning{ACCV-14 submission ID \ACCV14SubNumber}
\authorrunning{ACCV-14 submission ID \ACCV14SubNumber}

\author{Anonymous ACCV 2014 submission}
\institute{Paper ID \ACCV14SubNumber}

\maketitle

%===========================================================
\begin{abstract}
This paper describes a framework for modeling human actions incorporating surroundings cues. We consider a weak foreground-background segmentation approach in order to robustly extract not only human aligned features, but also global motion and context information. Our approach relies on the recently proposed Improved Trajectories to both separate and describe the foreground motion while computing SIFT and global motion on background feature points. Our experiments on four challenging benchmarks (HMDB51, Hollywood2, Olympic Sports, UCF50) show that our surrounding features provide significant performance improvements compared to state-of-the-art algorithms.
\end{abstract}

%===========================================================
\section{Introduction}
\label{introduction}

Human action recognition is a challenging task for computer vision algorithms due to the large variabilities in video data caused by occlusions,
camera motions, actor and scene appearances, among others. A popular current trend in action recognition methods relies on using local video
descriptors to represent visual events in videos \cite{Dollar2005, Laptev2005, WangCVPR2011}. These features are usually aggregated into
a compact representation, most commonly into a bag-of-features (BoF) representation framework \cite{Schuldt2004}. The advantage of this
simple representation is that it avoids difficult pre-processing steps such as motion segmentation and tracking.
In the BoF representation, local descriptors are quantized using a pre-computed codebook of visual patterns. This representation combined
with discriminative classifiers such as support vector machines (SVM), has achieved tremendous success in action recognition in controlled
scenarios \cite{Blank2005,Schuldt2004}. Due to its simplicity, BoF requires the use of strong, robust and informative features, which can be 
obtained reliably in such simplified scenarios. However, recent efforts in the collection of more realistic datasets from movies and web site 
\cite{Kuehne2011,Marszalek2009,Rodriguez2008} represent a challenge for existing methods due to dynamic backgrounds, changes in light 
conditions and camera motions among other noisy conditions. 


\section{Preliminaries}
\label{preliminaries}
\subsection{Camera compensation using Fundamental Matrix}
\subsection{Foreground/Background trajectories}

\section{Contextual features}
\label{scene}
This section describes our methodology for capturing background information or \textit{contextual features}. We present a novel approach for including description of background trajectories. We argue that combining a pure foreground description \cite{wang2013} with additional surrounding cues have a significant contribution to actions description. To obtain these background trajectories, we perform a weak foreground-background separation thresholding trajectory displacement. Then, we explicitly model the global motion in the video and the context appearance using those background feature points. 
%It allow us to capture information about the context of the action. Specifically, we encode how the camera moves on the video and also encode the appearance of the scenario where the action is executed. Finally, this new two set of features are added to our fully representation, \ie foreground features \cite{wang2013} and surrounding features.

\subsection{Foreground-background segmentation}
In order to recover information related with actions context, we need separate track features as foreground or background. A lot of information related with the actor are included on the Improved Trajectories approach. It results beneficial when capturing the spatio-temporal appearance of human actions. However, we claim that modeling contextual information needs to be performed on background feature points. We apply a simple strategy to weakly labeling trajectory features. We compute trajectories as described in \cite{wang2013}, but also keep in a separate group the trajectories which has a lower displacement than a threshold $\alpha$. This allows features points be classified as foreground or background.

Figure \ref{fig:trajectory-segmentation} illustrates weak trajectory segmentation results in different video sequences. Red and blue color represent feature points associated with the foreground and background features respectively. We notice that this simple approach allow us to weakly separate between the action trajectories and the background. 

\begin{figure*}[t!]
\begin{center}
%\fbox{\rule{0pt}{3in} \rule{0.9\linewidth}{0pt}}
\includegraphics[width=0.98\linewidth]{fig/segmentation.png}
\end{center}
\caption{Illustration of weak segmentation of feature points. }
\label{fig:trajectory-segmentation}
\end{figure*}

\subsection{Camera motion}
Since videos are normally filmed with an intention, the camera motion is a helpful component to make a better description of human actions. Recent approaches relies on canceling the camera motion in order to capture a more pure action cues. In contrast to most existing works, we employ a low level feature for capturing the global motion in the video. 
\subsection{Context appearance}
Human actions could be recognized by a set of cues. Beyond the movements, the scenario where action is executed is a critical component to recognize actions. For example, springboard can be only executed if there is a pool where submerge. It motivate us to encode visual appearance of the scene. This context appearance is encoded computing SIFT \cite{lowe2004} descriptors around the trajectory points associated with the background. We compute SIFT features in dense fashion and then we filter out those feature points that fall within our foreground mask. Context appearance focuses more on the scenario itself, as observed in Figure \ref{fig:surrounding-features}. 

\begin{figure*}[t!]
\begin{center}
\fbox{\rule{0pt}{1in} \rule{0.9\linewidth}{0pt}}
\end{center}
\caption{Illustration of weak segmentation of feature points. }
\label{fig:surrounding-features}
\end{figure*}
\section{Experimental results}
\label{results}
\subsection{Datasets and evaluation protocol}
\label{subsec:datasets}
We use four public datasets \cite{kuehne2011, marszalek2009, niebles2010, reddy2013} and their corresponding evaluation protocols. In this section we briefly describe each dataset.

\textbf{HMDB51} \cite{kuehne2011} includes a large collection of human activities categorized on 51 classes. It collects 6766 videos from different media resources \ie digitized movies, public databases and user generated web video data. Due to a large amount of videos contains undesired camera motions, the authors provide a stabilized version of the dataset. However, since we look at the camera motion as an informative cue, non-stabilized version of the dataset is used. For evaluating performance, we adopt the same protocol proposed by the dataset authors \ie computing the mean accuracy under three fixed train/test splits.

\textbf{Hollywood2} \cite{marszalek2009} contains a wide number of videos retrieved from 69 different Hollywood movies. It is divided in 12 categories including short actions such as Kiss, Answer Phone and Stand Up. This dataset remains as one of the most challenging despite the small number of action classes. Change of camera view,  camera motion and unchoreographed execution introduces more difficult a the time of recognition. To evaluate performance, we follow the author's protocol where videos are separated in two different sets: a training set of 823 videos and a testing set of 884 videos. We use training videos to learn our action models and then compute the mean average precision (mAP) over all action classes.

\textbf{Olympic Sports} \cite{niebles2010} or \textbf\textit{{Olympic}} comprises a set of 783 sport related YouTube videos. This set of videos are semi-automatically labeled using Amazon Mechanical Turk. This dataset establish new challenges for recognition because of it jumps from simple actions (\eg Kiss) to complex actions (\eg Hammer throw). All of these complex actions are related with olympic sports including actions like \textit{Long jump}, \textit{Pole vault} and \textit{Javelin throw}. As proposed by the author's dataset, we measure performance calculating the mAP over all dataset categories.

\textbf{UCF50} \cite{reddy2013} includes 6618 videos of 50 different human actions.  This dataset presents several recognition challenges due to large variations in camera motion, cluttered background, viewpoint, etc. Action categories are grouped into 25 sets, where each set consists of more than 4 action clips. Recognition performance is measured by applying a leave-one-group-out cross-validation and average accuracy over all group splits is reported. 

%%%%%%%%%%%%%%%%% Figure: Effect of sampling %%%%%%%%%%%%%%%%%%%
\begin{figure*}[t!]
\begin{center}
%\fbox{\rule{0pt}{1.2in} \rule{0.9\linewidth}{0pt}}
\includegraphics[width=0.98\linewidth]{fig/sampling.png}
\end{center}
\caption{Effect of feature sub-sampling when generating codebook.}
\label{fig:feature_sampling}
\end{figure*}
%%%%%%%%%%%%%%%%%%%%%%%%%%%%%%%%%%%%%%%%%%%%%%%%%%%%%%%%%%%%%%%%

\subsection{Impact of contextual features}
We conduct further experiments to measure the  contribution of our contextual features. Our Baseline corresponds to using only Foreground features for describing actions. Per-descriptor performances are compared to that established baseline. Also, we investigate the effect of combining contextual features with Foreground cues. As well, contextual features performance is evaluated under two action recognition representations \ie Bag of Features and Fisher vectors. Below, we present an analysis of obtained results.\\\\
\textbf{Representation}. As suggested in recent works \cite{perronnin2010, wang2013, xwang2013} Fisher vectors provides a boosted performance compared to traditional Bag of Feature representations. We found in our experiments that Fisher vectors also boost our contextual descriptors performance, as presented in Table \ref{tab:features}. However, we note that using Fisher vectors is less important with our CamMotion descriptor due to its low dimensionality. Even so, Fisher vectors are used for following analysis.\\\\
\textbf{Foreground-background}. As described in Section \ref{scene}, we perform a weak separation between background and foreground feature points. We measure the effect on performance of this separation in our contextual features. We note that this type of weak segmentation provides a significant boost in performance, as observed in Table \ref{tab:segmentation}. When feature points are localized on the background, SIFT features focuses on the scene appearance avoiding information of actors and foreground objects. The gain over computing SIFT over all features points is as follow: +0.3\% for HMDB51, +4.2\% for Hollywood2, +5.2\% for Olympics and +3.9\% for UCF50. The same behavior is observed with our CamMotion descriptor. Performance is boosted in all datasets when Fundamental Matrix is computed based on background tracks.\\\\
\textbf{Context appearance}. While by itself SIFT achieves a discrete performance, it produces notable improvements when combined with foreground descriptors. As Table \ref{tab:features} reports, performance is significantly improved over all datasets. Interestingly, we note that SIFT descriptor produces higher improvements in HMDB51 and UCF50 \ie +2.7\% and +2.4\% respectively.\\\\
\textbf{Camera motion}. Experiment results evidences that action recognition is noticeably improved when a global motion is incorporated to Foreground features. Our CamMotion provides slightly lower contributions in performance than the SIFT descriptor. We observe a significantly contribution over all datasets except on HMDB51 where recognition performance decrease. We attribute this negative effect to that HMDB51 presents several videos containing a shaking camera motion over all classes. This unable our CamMotion to capture discriminative cues over action categories. 

%%%%%%%%%%%%% Table: Effect of selected points %%%%%%%%%%%%%
\begin{table*}
\caption{Effect of separating background feature points on contextual features. Experimental results consistently show that contextual features are better captured on background regions. As observed, SIFT and CamMotion descriptors tend to be more 
discriminative when they are extracted from non-foreground feature points.}
\begin{center}
{
\begin{tabular}{|c|c c|c c c c|}
\hline
& \multicolumn{2}{|c|}{Feature points} & \multicolumn{4}{|c|}{Datasets} \\
Feature $\downarrow$ & Foreground & Background & HMDB51 & Hollywood2 & Olympics & UCF50 \\
\hline
SIFT & \checkmark & & 19.5\% & 22.1\% & 33.5\% & 44.7\% \\
SIFT & & \checkmark & \textbf{20.1\%} & \textbf{28.5\%} & \textbf{39.6\%} & \textbf{49.8\%} \\
SIFT & \checkmark & \checkmark & 19.8\% & 24.3\% & 34.4\% & 45.9\% \\
\hline
CamMotion & \checkmark & & 9.7\% & 14.9\% & 19.5\% & 13.7\% \\
CamMotion & & \checkmark & \textbf{14.1\%} & \textbf{22.1\%} & \textbf{27.2\%} & \textbf{19.5\%} \\
CamMotion & \checkmark & \checkmark & 12.9\% & 18.7\% & 21.8\% & 17.2\% \\
\hline
\end{tabular}
}
\end{center}
\label{tab:segmentation}
\end{table*}
%%%%%%%%%%%%%%%%%%%%%%%%%%%%%%%%%%%%%%%%%%%%%%%%%%%%%%%%%%%%

%%%%%%%%%%%%%% Table: Feature analysis %%%%%%%%%%%%%%%%%%%%%
\begin{table*}
\caption{Impact of our contextual features in recognition performance. Bag of Features generally performs poor than Fisher vectors. Both SIFT and CamMotion show important improvements in performance when they are combined with foreground descriptors.}
\begin{center}
{
\def\arraystretch{1.11}
\setlength{\tabcolsep}{3.66pt}
\begin{tabular}{ |c c c|c c c c| }
\hline
\multicolumn{3}{|c|}{Features} & \multicolumn{4}{|c|}{Datasets} \\
Foreground & SIFT & CamMotion & HMDB51 & Hollywood2 & Olympics & UCF50 \\
\hline
\multicolumn{7}{|c|}{Framework: Bag of Features} \\
\hline
\checkmark & & & 51.2\% & 60.1\% & 79.8\% & 85.9\% \\
& \checkmark & & 19.5\% & 28.7\% & 36.4\% & 45.7\% \\
& & \checkmark & 13.5\% & 21.8\% & 26.9\% & 19.3\% \\
\checkmark & \checkmark & & 53.8\% & 60.9\% & 81.1\% & 87.2\% \\
\checkmark &  & \checkmark & 50.9\% & 60.4\% & 80.6\% & 86.8\% \\
& \checkmark & \checkmark & 20.7\% & 36.2\% & 43.7\% & 50.3\% \\
\checkmark & \checkmark & \checkmark & 51.7\% & 61.6\% & 81.7\% & 87.6\% \\
\hline
\multicolumn{7}{|c|}{Framework: Fisher vectors} \\
\hline
\checkmark & & & 56.5\% & 62.4\% & 90.4\% & 90.9\% \\
& \checkmark & & 20.1\% & 28.5\% & 39.6\% & 49.8\% \\
& & \checkmark & 14.1\% & 22.1\% & 27.2\% & 19.5\% \\
\checkmark & \checkmark & & \textbf{59.2\%} & \textbf{63.5\%} & \textbf{91.6\%} & \textbf{93.3\%} \\
\checkmark &  & \checkmark & 55.9\% & 62.9\% & 91.3\% & 93.1\% \\
& \checkmark & \checkmark & 22.3\% & 36.5\% & 46.5\% & 54.3\% \\
\checkmark & \checkmark & \checkmark & \textbf{57.9\%} & \textbf{64.1\%} & \textbf{92.5\%} & \textbf{93.8\%} \\
\hline
\end{tabular}
}
\end{center}
\label{tab:features}
\end{table*}
%%%%%%%%%%%%%%%%%%%%%%%%%%%%%%%%%%%%%%%%%%%%%%%%%%%%%%%%%%%%%


\subsection{Comparison with the state of the art}
We set side by side our method with recent methods that address the same application using similar representations, \ie methods that use dense trajectory points to represent video sequences \cite{wang2013, jiang2012, jain2013} in Table \ref{tab:stateofart}. We also present results for our own implementation of \cite{wang2013}, which correspond to our baseline (Foreground). The gain over the recent paper \cite{wang2013}, which reports the best performance in the literature, is as follow: \textbf{+2\%} for HMDB51, \textbf{+1.4\%} for Olympic Sports and \textbf{2.6\%} for UCF50. We also achieve a comparable performance on Hollywood2 dataset with only 0.2\% less in the mAP value. Since Human Detection (HD) is not included in our trajectory extraction stage, a more direct comparison its the non-HD approach of Wang \etal \cite{wang2013}. In that case, our method outperforms their improved trajectories in \textbf{3.3\%} for HMDB51, \textbf{1.1\%} for Hollywood2, \textbf{2.3\%} for Olympic Sports and \textbf{3.3\%} for UCF50.

%%%%%%%%%%%%%% Table: State-of-the-art %%%%%%%%%%%%%%%%%%%%%%
\begin{table*}
\caption{Comparison with the state-of-the-art on challenging datasets. Our method improves reported results in the state-of-the-art for three different datasets, HMDB51, Olympic Sports and UCF50 and obtains competitive peformance in Hollywod2.}
\begin{center}
{
\begin{tabular}{ |l| c c c c| }
\hline
Approach $\downarrow$ & HMDB51 & Hollywood2 & Olympics & UCF50 \\
\hline
Jiang \etal \cite{jiang2012} & 40.7\% & 59.5\% & 80.6 & - \\
Jain \etal \cite{jain2013} & 52.1\% & 62.5\% & 83.2 & - \\
Wang \etal \cite{wang2013} non-HD & 55.9\% & 63.0\% & 90.2\% & 90.5\% \\
Wang \etal \cite{wang2013} HD & 57.2\% & \textbf{64.3\%} & 91.1\% & 91.2\% \\
\hline
\multicolumn{5}{|c|}{\textit{Our methods with Fisher vectors}} \\
\hline
Baseline (Foreground) & 56.5\% & 62.4\% & 90.4\% & 90.9\% \\
Foreground + SIFT & \textbf{59.2\%} & 63.5\% & \textbf{91.6\%} & \textbf{93.3\%} \\
Foreground + SIFT + CamMotion  & \textbf{57.9\%} & \textbf{64.1\%} & \textbf{92.5\%} & \textbf{93.8\%} \\
\hline
\end{tabular}
}
\end{center}
\label{tab:stateofart}
\end{table*}
%%%%%%%%%%%%%%%%%%%%%%%%%%%%%%%%%%%%%%%%%%%%%%%%%%%%%%%%%%%%%

\section{Conclusion}
\label{conclusion}

In this paper, we describe the use of camera movement and surrounding scene appearance as cues for action recognition.
Our method is the first to incorporate information about global camera motion as a feature for recognition,
which is shown experimentally to enhance recognition performance in multiple benchmarks. We have also
revisited the use of scene appearance as a feature for action recognition. We show that even a weak foreground/background
segmentation can provide more reliable features related to the surrounding scene compared to the previous
standard practice of using features from the entire frame. We highlight that the use of our camera motion and scene appearance
features in combination with standard motion features that focus on the foreground regions
can lead to state-of-the-art performances when coupled with the simplest classifiers
and encodings used in the current literature.
We expect that more sophisticated modeling frameworks would also be able to leverage our proposed feature extraction
process to further improve the current action recognition performance.

%here would be your acknowledgement (if any) in the final accepted paper

%===========================================================
\bibliographystyle{splncs}
\bibliography{egbib}

%this would normally be the end of your paper, but you may also have an appendix
%within the given limit of number of pages
\end{document}